\documentclass[11pt,a4paper,twocolumn,twoside]{paper}
\usepackage[utf8]{inputenc}
\usepackage{amsmath}
\usepackage{amsfonts}
\usepackage{amssymb}
\usepackage{graphicx}
\usepackage{listings}
\usepackage{wrapfig}
\usepackage{enumitem}
\usepackage{helvet}
\renewcommand{\familydefault}{\sfdefault}
\usepackage[left=1.5cm, right=1.5cm, top=3cm, bottom=2cm]{geometry}
\author{Guillaume Quint (577923)}
\title{\vspace{-4em}Documentazione Progetto Reti Informatiche\\Ingegneria Informatica UNIPI A.A. 21/22}
\subtitle{\today}
\begin{document}
\maketitle

\paragraph{Descrizione}
	L'applicazione distribuita di messaggistica realizza tutti i
    requisiti funzionali descritti nella documentazione fornita dal
    professore. Segue il paradigma client-server per la registrazione,
    autenticazione, controllo della connettività, inoltro dei messaggi e
    salvataggio in cache per la messaggistica offline. Inoltre svolge
    una comunicazione di tipo peer-to-peer per la funzionalità di
    file-sharing e nella comunicazione tra devices nel caso in cui il
    server fosse irraggiungibile. Quest'ultima funzionalità di tipo
    inaffidabile è stata aggiunta rispetto alle specifiche originali, ed
    è descritta nel paragrafo del device.
\paragraph{Strutture dati utilizzate}
	L'unica struttura dati utilizzata è la single-linked-list con
    puntatori alla testa e alla coda per facilitarne l'operazione di
    append. La lista viene utilizzata per la gestione della rubrica da
    parte dei devices e per la gestione dei profili attivi sul server.

    L'utilizzo delle liste ha semplificato il lavoro in fase di
    sviluppo, ed è relativamente efficiente rispetto ad altre strutture
    dati più complesse nelle condizioni testate (non più di 10 utenti).
    D'altra parte, la complessità lineare di ricerca nella lista
    richiederebbe la sostituzione con vettori dinamici o hash-tables nel
    caso in cui il volume di utenza aumentasse notevolmente.
\paragraph{Modalità verbose}
    Per attivare la modalità verbose è necessario cambiare nelle prime
    linee dei file \texttt{server.c}, \texttt{device.c},
    \texttt{util/IOMultiplex.c} e/o \texttt{util/network.c} la
    definizione del flag per il compilatore da \texttt{DEBUG\_OFF} a
    \texttt{DEBUG\_ON}, così che venga definita la macro
    \texttt{DEBUG\_PRINT}

    Per evitare che la routine periodica di richiesta di heartbeat da
    parte del server "inondi" l'interfaccia dei comandi, la modalità
    verbose è disattivata per il modulo network. Attivandola, è
    possibile vedere ogni singolo pacchetto inviato e ricevuto da client
    e server.
\paragraph{Server}
 All'avvio del server, viene mostrato il numero di porta su cui
    questo è in ascolto, insieme ad una breve descrizione dei comandi
    disponibili.

    \texttt{help} mostra la lista dei comandi, \texttt{list} mostra la
    lista degli utenti attualmente connessi nel formato
    \texttt{username*timestamp\_login*porta} ed \texttt{esc} chiude il
    server senza avvisare i devices

    La registrazione e l'autenticazione degli utenti viene svolta
    appoggiandosi ad uno specifico file, chiamato \texttt{shadow}, con
    il ruolo di database. Ogni linea del file equivale ad un record per
    un corrispettivo utente registrato. I campi, in formato ASCII
    separati da spazio, sono nell'ordine
    \texttt{username\ passowrd\ porta\ timestamp\_login\ timestamp\_logout\ chances}
\begin{itemize}[leftmargin=4mm, noitemsep]


	\item La porta, se diversa da -1, identifica la porta sul localhost del
      device su cui è in ascolto l'utente che ha effettuato il login

      \item timestamp\_login e timestamp\_logout sono l'epoch time in
      secondi degli ultimi eventi di login e logout, e permettono di
      sapere (se ts\_login \textgreater{} ts\_logout) se l'utente è
      attualmente connesso

      \item Il numero di chances viene utilizzato dalla routine di controllo
      connettività dei devices per tenere traccia di quanto sia
      probabile che un utente si sia disconnesso senza avvisare. La
      richiesta di \emph{heartbeat} (messaggio con codice
      \texttt{HRTBT}), infatti, viene svolta sul protocollo UDP per
      minimizzare il traffico, ma la natura inaffidabile del protocollo
      non assicura che tale richiesta arrivi correttamente. Per evitare
      che un device venga considerato offline ingiustamente, è
      necessario che non risponda per un numero \texttt{MAX\_CHANCES}
      (di default, 3) di volte senza comunicare al server un messaggio
      con codice \texttt{ALIVE}, il quale è sufficiente a ripristinare
      il numero massimo di chances per quell'utente.
\end{itemize}
    Il server ha il ruolo di inoltrare i messaggi ai destinatari per
    conto dei vari utenti. Se tutti i destinatari sono online e ricevono
    il messaggio, al mittente viene inviata una risposta con il codice
    \texttt{READ:} a significare che il suo messaggio è stato
    letto da tutti. Altrimenti la risposta ha il codice \texttt{CACHE} e
    per ogni utente che non ha ricevuto il messaggio, il server lo salva
    localmente. 
    
    In particolare, il contenuto del messaggio verrà aggiunto ad un file
    chiamato come il mittente che si trova all'interno della cartella 
    \texttt{hanging} e della sottocartella chiamata come il destinatario
	(in breve, il messaggio andrà nel file 
	\texttt{./hanging/\textless{}destinatario\textgreater{}/\textless{}mittente\textgreater{}}).

    Quando il server riceve un comando di \texttt{hanging} (messaggio
    con codice \texttt{HANG?}) da parte di un utente, gli basta
    rispondere con il contenuto della cartella
    \texttt{./hanging/\textless{}utente\textgreater{}/}

    Il comando \texttt{show} corrisponde al trasferimento del contenuto
    del file
    \texttt{./hanging/\textless{}destinatario\textgreater{}/\textless{}mittente\textgreater{}}
    e successiva sua eliminazione. Inoltre, l'evento di lettura dei
    messaggi non recapitati viene segnalato all'utente mittente,
    affinché possa sincronizzare l'informazione di lettura con il suo
    storico di messaggi salvato localmente.

    Infine il server sfrutta la funzionalità di timeout della primitiva
    \texttt{select} per effettuare una routine di controllo in maniera
    periodica. In particolare invia ad ogni device attualmente online
    una richiesta di heartbeat, affinché questi confermino il loro stato
    di connettività. All'arrivo di un segnale di alive da parte di un
    device, il server ripristina le chances per quell'utente.
\paragraph{Device}
	All'avvio di un device viene mostrata la porta su cui questo è in
    ascolto e una breve lista dei comandi disponibili. Inizialmente
    nessun utente è connesso. E' possibile accedere o registrarsi con i
    rispettivi comandi \texttt{in} e \texttt{signin} specificando le
    proprie credenziali. Alla creazione di un nuovo profilo viene creata
    la propria cartella personale del device (con il nome dell'utente
    nel current path) e il proprio spazio dove ospitare i messaggi non
    recapitati quando l'utente è offline sul server (con il nome
    dell'utente nella cartella \texttt{./hanging/}). Attenzione: NON
    viene creato il file di rubrica. Inoltre l'applicazione non
    implementa alcuna funzione per crearlo e modificarlo. Pertanto il
    nuovo utente non sarà in grado di comunicare con nessuno, fintanto
    che non verrà creato il file \texttt{contacts} nella cartella
    personale (quindi
    \texttt{./\textless{}nome\_utente\textgreater{}/contacts})
    contenente il nome degli utenti con cui si vuole chattare, uno per
    riga. La corretta registrazione di un utente viene seguita da un
    login automatico.

    Di default (ed il comando \texttt{make\ clean} riporta a questo
    stato) sono già presenti tre utenti. Le credenziali e le rispettive
    rubriche sono:
    \begin{itemize}[leftmargin=4mm, noitemsep]
      \item\texttt{user1\ pass} ha tra i contatti \texttt{user2}
      \item\texttt{user2\ pass} ha tra i contatti \texttt{user1} e \texttt{user3}
      \item\texttt{user3\ pass} ha tra i contatti \texttt{user2}
    \end{itemize}
    Successivamente all'accesso di un utente, il device controlla la
    presenza di un file di rubrica nella cartella personale e, se lo
    trova, ne carica il contenuto. Inoltre, se trova un file di
    disconnessione non notificata al server, lo legge e tenta di
    comunicarlo; se ci riesce, elimina il file.

    Quando un utente è connesso su un device dispone di una lista di
    comandi aggiuntivi. \texttt{hanging} chiede al server di accedere al
    proprio spazio personale di messaggi non ricevuti
    (mostra il contenuto della cartella
    \texttt{./hanging/\textless{}nome\_utente\textgreater{}/}). I files
    che vi si trovano, chiamati come il nome dell'utente mittente,
    contengono i messaggi che non sono stati ancora letti. Tramite il
    comando \texttt{show} (invio al server di un messaggio con codice
    \texttt{SHOW:}) su uno di quei nomi, l'utente legge i messaggi che
    non gli erano stati recapitati e il device li appende allo storico
    dei messaggi nella chat con quell'utente.

	I comandi \texttt{out} e \texttt{esc} effettuano la disconnessione
	dell'utente dal device, cercando di comunicare l'uscita al server.
	Se quest'ultimo fosse irraggiungibile, il device salva localmente
	sul file \texttt{./\textless{}nome\_utente\textgreater{}/disconnect}
	l'istante di disconnessione, così che possa tentare di recapitarlo
	al prossimo login.

    \texttt{chat} permette di entrare nella chat con un utente presente
    nella rubrica, a prescindere che questo sia connesso. Viene quindi
    mostrato lo storico dei messaggi che coinvolgono questo utente e,
    dopo un separatore, quelli che gli sono stati inviati ma non
    recapitati, quindi che sono salvati sul server in attesa di essere
    letti. Quando il server notificherà della lettura di quei messaggi,
    verranno automaticamente spostati nello storico dei messaggi letti.

    Il prompt dei comandi si modifica se si sta chattando con qualcuno
    per mostrare il nome degli utenti destinatari.

    Tramite il comando \texttt{\textbackslash{}q} si esce dalla chat. Il
    comando \texttt{\textbackslash{}u} mostra lo status di tutti gli
    utenti presenti in rubrica. In particolare il device esegue una
    richiesta \texttt{ISONL} sul server, affinché questi gli comunichi
    se un utente è connesso (risposta con codice \texttt{ONLIN}
    contenente la porta su cui l'utente è raggiungibile), disconnesso
    (risposta con codice \texttt{DSCNT}) o sconosciuto, ossia non
    registrato sul server (risposta con codice \texttt{UNKWN}). Tramite
    il comando \texttt{\textbackslash{}a} è possibile aggiungere un
    utente alla lista dei destinatari, solo se questo è attualmente
    connesso. Infine il comando \texttt{share} permette di condividere
    un file con tutti i destinatari. La comunicazione non coinvolge il
    server, ma se il device non dovesse conoscere la porta su cui
    raggiungere un destinatario, effettuerà prima una richiesta di
    risoluzione della porta sul server. Per evitare di bloccare l'input
    in attesa del trasferimento di files di grosse dimensioni, la
    condivisione viene effettuata su un nuovo processo per ogni
    destinatario. Alla ricezione di un file da parte di un altro device,
    questo viene salvato nella propria cartella con il nome
    \texttt{copy-\textless{}nome\_originale\textgreater{}}

    All'interno della chat è sufficiente digitare del testo seguito da
    invio per inviare una richiesta di inoltro del messaggio al server
    per tutti i destinatari della chat. Se l'inoltro va a buon fine per
    tutti i destinatari, il messaggio viene aggiunto allo storico dei
    messaggi di ognuno dei riceventi (un file nella cartella personale
    con il nome del destinatario
    \texttt{./\textless{}nome\_utente\textgreater{}/\textless{}destinatario\_chat\textgreater{}}),
    altrimenti viene salvato in un file di cache dei messaggi inviati e
    non letti (file nella cartella personale nella forma
    \texttt{./\textless{}nome\_utente\textgreater{}/chached-\textless{}destinatario\_chat\textgreater{}},
    in attesa della notifica di lettura da parte del server

    Nel caso in cui il server non fosse raggiungibile per l'inoltro del
    messaggio, il device cercherà di comunicare direttamente con gli altri
    utenti tramite le informazioni che ha a disposizione. Fallisce nel
    caso in cui non abbia salvato precedentemente la porta del destinatario, 
    in quanto non può fare affidamento sul server per ottenere questa informazione.
\paragraph{Gestione del IO}
Tramite la primitiva \texttt{select}, l'applicazione sfrutta il
    multiplexing dell'IO con possibilità di timeout per gestire i
    comandi provenienti da tastiera e dai sockets senza effettuare
    un'attesa attiva. La gestione di richieste provenienti sul
    protocollo TCP viene svolta su un processo separato, così che non si
    rischi di bloccare l'intera applicazione.
\paragraph{Network utility}
	L'applicazione fa affidamento sia sul protocollo TCP per la
    comunicazione affidabile dei messaggi, sia sul protocollo UDP per
    l'invio di segnali di notifiche relativamente allo stato di
    connessione dei devices. Tutte le comunicazioni seguono lo schema
    \texttt{CODICE-LUNGHEZZA-MESSAGGIO}. Viene prima di tutto inviata
    una stringa da 5 caratteri(+1 null byte) che definisce la natura del
    pacchetto, successivamente viene trasferito un valore numerico pari
    alla lunghezza del messaggio in byte ed infine, se la lunghezza è
    diversa da zero, viene inviato il messaggio vero e proprio.
\end{document}
